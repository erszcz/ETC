\documentclass{article}

% a bunch of useful packages.
\usepackage[inference]{semantic}
\usepackage{amsmath} \allowdisplaybreaks % lets align equations break over pages.
\usepackage{hyperref}
\usepackage[amsmath,hyperref,amsthm]{ntheorem}
\usepackage{thmtools}
\usepackage[capitalize]{cleveref}
\usepackage{mathpartir}
\usepackage{stmaryrd}
\usepackage{amssymb}
\usepackage{latexsym}
\usepackage{color}
\usepackage[usenames,dvipsnames]{xcolor} % the best way to colour text
\usepackage[colorinlistoftodos]{todonotes} 
\usepackage{tikz} \usetikzlibrary{positioning,shadows,arrows,calc,backgrounds,fit,shapes,snakes,shapes.multipart,decorations.pathreplacing,shapes.misc,patterns}
\usepackage{xspace}
\usepackage{scalerel}
\usepackage{bm} %bold math. a mess to use.
\usepackage{bussproofs} % for logic-style proofs

\hypersetup{ pdfpagemode=UseOutlines, colorlinks=true, linkcolor=red, citecolor=blue }

% organise your code.
\input{cmds}

\title{
	Typesetting $\lambda$-calculi with \LaTeX
}
\author{
	Marco Patrignani
}
\date{}

%%%%%%%%%%%%%%%%%%%%%%%%%%%%%%%%%%%%%%%%%%%%%%%%%%%%%%%%%%%%%%%%%%%%%%%%%%%%%%%%%%%%%%%%%%%%%%%%%%%%%%%%%%%%%%%%%%%%%%%%%%%%%%%%%%%%%%%%%%%%%%%%%%%%%%%%%%%%%%%%%%%%%%%%%%%%%%%%%
%%%%%%%%%%%%%%%%%%%%%%%%%%%%%%%%%%%%%%%%%%%%%%%%%%%%%%%%%%%							DOCUMENT						  %%%%%%%%%%%%%%%%%%%%%%%%%%%%%%%%%%%%%%%%%%%%%%%%%%%%%%%%%%%
%%%%%%%%%%%%%%%%%%%%%%%%%%%%%%%%%%%%%%%%%%%%%%%%%%%%%%%%%%%%%%%%%%%%%%%%%%%%%%%%%%%%%%%%%%%%%%%%%%%%%%%%%%%%%%%%%%%%%%%%%%%%%%%%%%%%%%%%%%%%%%%%%%%%%%%%%%%%%%%%%%%%%%%%%%%%%%%%%

\begin{document}
\maketitle

This document contains a reference of the syntax and semantics for ULC and STLC.

Syntax highlighting is nice way to guide you in visually separating each language; please inform me whether they cause any distraught.

\begin{center}
	\Large
	Please read the latex comments too.
\end{center}
\tableofcontents

\newpage

\section{\ulc: Untyped Lambda Calculus}

\subsection{Syntax}
% all the stuff here is wrapped in \com{}, which is the macro i use to typeset stuff in italics, black.
% removing that macro won't change anything visually (though if later on you want to typeset all ULC in, say, red, it's easier this way)
% unless you want to typeset coloured stuff, do not care for \com{}, \src{} and \oth{} commands

% \mid are vertical separators
% the align* environment will align on &, which are put after the ::=, with an extra \ for space
\begin{align*}
	\com{t} \bnfdef&\
		\com{n} \mid \com{\lam{x}{t}} \mid \com{x} \mid \com{t\ t} \mid \com{t \op t} % the \ is a single space
	\\
	\com{v} \bnfdef&\
		\com{n} \mid \com{\lam{x}{t}}
	\\
	\com{\Omega} \bnfdef&\
		\com{t} \mid \com{\fail}
	\\
	\com{\evalctx} \bnfdef&\
		\com{\hole{\cdot}} \mid \com{\evalctx\ t} \mid \com{(\lam{x}{t})\ \evalctx} \mid \com{\evalctx \op t} \mid \com{n \op \evalctx}
\end{align*}
$\op$ ranges over $+, -, *$.

\subsection{Structural Operational Semantics}

\subsubsection{Small Step, Call by Value (SSV)}
\begin{align*}
	\com{\Omega\red\Omega} 
	&&
	\text{ judgement }
\end{align*}
Reductions.
% inference rules are best put centered, with this environment
\begin{center}
	% this command is for generic inference rules. 
	% the 4th parameter is the label, which gets referred as    tr:ssv-beta    in the first case, i.e., always add a   tr:   to what is between brackets
	\typerule{SSV-Beta}{}{
		\com{(\lam{x}{t})~v \red t\subst{v}{x}}
	}{ssv-beta}
	\typerule{SSV-Op}{}{
		\com{n\op n' \red n\llbracket\op\rrbracket n'} 	% llbracket rrbracket should be made into a command. try it.
	}{ssv-op}
	\typerule{SSV-App1}{
		\com{t_1\red t_1'}
	}{
		\com{t_1\ t_2 \red t_1' t_2}
	}{ssv-app1}
	\typerule{SSV-App2}{
		\com{t_2\red t_2'}
	}{
		\com{(\lam{x}{t})\ t_2 \red (\lam{x}{t})\ t_2'}
	}{ssv-app2}
	\typerule{SSV-Op1}{
		\com{t_1\red t_1'}
	}{
		\com{t_1\op t_2 \red t_1'\op t_2}
	}{ssv-op1}
	\typerule{SSV-Op2}{
		\com{t_2\red t_2'}
	}{
		\com{n\op t_2 \red n\op t_2'}
	}{ssv-op2}
\end{center}
Fail reductions.
\begin{center}
	\typerule{SSV-Op-fail-l}{}{
		\com{(\lam{x}{t})\op t \red\fail}
	}{ssv-f-opl}
	\typerule{SSV-Op-fail-r}{}{
		\com{n\ \op(\lam{x}{t}) \red\fail}
	}{ssv-f-opr}
	\typerule{SSV-App-fail-fun}{}{
		\com{n\ t\red\fail}
	}{ssv-f-fun}
	\typerule{SSV-App-fail-arg}{
		\com{t_2\red\fail}
	}{
		\com{(\lam{x}{t})\ t_2\red\fail}
	}{ssv-f-arg}
	\typerule{SSV-App-fail-1}{
		\com{t_1\red\fail}
	}{
		\com{t_1\ t_2\red\fail}
	}{ssv-f-app}
	\typerule{SSV-Op-fail-1}{
		\com{t_1\red\fail}
	}{
		\com{t_1\op t_2\red\fail}
	}{ssv-f-op1}
	\typerule{SSV-Op-fail-2}{
		\com{t_2\red\fail}
	}{
		\com{n\op t_2\red\fail}
	}{ssv-f-op2}
\end{center}

\subsubsection{Small Step, Call by Name (SSN)}
Remove \Cref{tr:ssv-app2} and replace \Cref{tr:ssv-beta} with:
\begin{center}
	\typerule{SSN-Beta}{}{
		\com{(\lam{x}{t})~t' \red t\subst{t'}{x}}
	}{ssn-beta}
\end{center}

\subsubsection{Big Step, Call by Value (SBV)}
\begin{align*}
	\com{\Omega\bigs\Omega} 
	&&
	\text{ judgement }
\end{align*}
Reductions.
Reductions for failing are omitted.
\begin{center}
	\typerule{SBV-val}{}{
		\com{v \bigs v}
	}{sbv-val}
	\typerule{SBV-op}{
		\com{t \bigs n }
		&
		\com{t' \bigs n'}
	}{
		\com{t \op t' \bigs n \llbracket \op \rrbracket n'}
	}{sbv-op}
	\typerule{SBV-app}{
		\com{t \bigs \lam{x}{t''}}
		&
		\com{t' \bigs v}
		&
		\com{t''\subst{v}{x} \bigs v'}
	}{
		\com{t\ t' \bigs v'}
	}{sbv-app}
\end{center}

\subsection{Contextual Operational Semantics, Call by Value (CSV)}
\begin{align*}
	\com{\Omega\cred\Omega} 
	&&
	\text{ judgement }
	\\
	\com{\Omega\credp\Omega} 
	&&
	\text{ judgement }
\end{align*}
Reductions.
Reductions for failing are omitted.
\begin{center}
	\typerule{CSV-ctx}{
		\com{t \credp t'}
	}{
		\com{\evalctx\hole{t} \cred \evalctx\hole{t'}}
	}{csv-ctx}
\end{center}
Plus \Cref{tr:ssv-beta,tr:ssv-op}, changing $\red$ with $\credp$. 

\section{\stlc: Simply-Typed Lambda Calculus}
\subsection{Syntax}
No changes save for program state.
\begin{align*}
	\src{\Omega} \bnfdef&\
		\src{t}
	\\
	\src{\Gamma} \bnfdef&\
		\srce \mid \src{\Gamma},(\src{x:\tau})
	\\
	\src{\tau} \bnfdef&\
		\src{N} \mid \src{\tau\to\tau}
\end{align*}

\subsection{Static Semantics}
\begin{align*}
	\src{\Gamma}\vdash\src{t}:\src{\tau} 
	&&
	\text{ judgement }
\end{align*}
Reductions.
\begin{center}
	\typerule{Type-var}{
		\src{x:\tau}\in\src{\Gamma}
	}{
		\src{\Gamma}\vdash\src{x}:\src{\tau}
	}{t-stlc-var}
	\typerule{Type-lam}{
		\src{\Gamma},\src{x:\tau}\vdash\src{t}:\src{\tau'}
	}{
		\src{\Gamma}\vdash\src{\lam{x:\tau}{t}}:\src{\tau\to\tau'}
	}{t-stlc-lam}
	\typerule{Type-num}{
	}{
		\src{\Gamma}\vdash\src{n}:\src{N}
	}{t-stlc-num}
	\typerule{Type-app}{
		\src{\Gamma}\vdash\src{t}:\src{\tau'\to\tau}
		&
		\src{\Gamma}\vdash\src{t'}:\src{\tau'}
	}{
		\src{\Gamma}\vdash\src{t\ t'}:\src{\tau}
	}{t-stlc-app}
	\typerule{Type-op}{
		\src{\Gamma}\vdash\src{t}:\src{N}
		&
		\src{\Gamma}\vdash\src{t'}:\src{N}
	}{
		\src{\Gamma}\vdash\src{t\ t'}:\src{N}
	}{t-stlc-op}
\end{center}

\subsection{Contextual Operational Semantics, Call by Value (CSV)}
No changes, repeated for clarity.
\begin{align*}
	\src{\Omega\cred\Omega} 
	&&
	\text{ judgement }
	\\
	\src{\Omega\credp\Omega} 
	&&
	\text{ judgement }
\end{align*}
Reductions.
No reductions for failing.
\begin{center}
	\typerule{CSV-ctx}{
		\src{t \credp t'}
	}{
		\src{\evalctx\hole{t} \cred \evalctx\hole{t'}}
	}{csv-stlc-ctx}
	\typerule{CSV-Beta}{}{
		\src{(\lam{x}{t})~v \credp t\subs{v}{x}}
	}{ssv-stlc-beta}
	\typerule{CSV-Op}{}{
		\src{n\op n' \credp n\llbracket\op\rrbracket n'}
	}{ssv-srlc-op}
\end{center}

\end{document}